\documentclass[10pt,a4paper]{article}
\usepackage{tabularray}
\usepackage{tikz}             
\usepackage{xeCJK}
\usepackage{amsthm}  
\usepackage[fontset=macnew]{ctex}
\usepackage{sgame}
\usepackage{geometry}
\usepackage{amsmath}
\usepackage{amssymb}
\usepackage{graphicx}
\usepackage{sectsty}
\usepackage{color}
\geometry{left=2.cm,right=2.cm,top=3.18cm,bottom=3.18cm}
\usepackage{pgfplots} % package used to implement the plot
\usepackage{pgf-pie}
\usepackage{exscale}
\usepackage{relsize}
\pgfplotsset{width=6.5cm, compat=1.6}
\usepackage{indentfirst}
\usepackage{float}
\usetikzlibrary{shapes,arrows}
\sectionfont{\color{blue}\selectfont}
\subsectionfont{\color{blue}\selectfont}
\subsubsectionfont{\color{blue}\selectfont}
\usepackage{hyperref}
\usepackage{marvosym}
\usepackage{color}
\usepackage{minted}
\usemintedstyle{xcode}
\usepackage{memorygraphs}
\usetikzlibrary{calc,shapes.multipart,chains,arrows.meta}
\usepackage{caption}
\usepackage{makecell}
\newtheorem{definition}{定义}
\newtheorem{theorem}{\textcolor{red}{定理}}
\newtheorem*{solution}{\kaishu 解}
\usepackage[skins]{tcolorbox} %必须标注skin,才能使用shadow命令显示阴影。
\tcbuselibrary{breakable} %breakable:支持跨页
\usepackage{fancyhdr} % 导入fancyhdr包
\pagestyle{fancy}
% 页眉设置
\fancyhead[L]{\textbf{龚舒凯\ 2022202790}}

\begin{document}
\title{{\Huge 数据结构与算法I实验报告{\large\linebreak\\}}{\huge 实验1:栈和队列的应用\linebreak\linebreak}}
\vspace{3cm}
%\author{\Large 龚舒凯\ 2022202790\ 应用经济-数据科学实验班}
\author{\\ \Large 龚舒凯\ 2022202790\ 应用经济-数据科学实验班\\
	\hfill\\
	\Large{\url{https://github.com/GONGSHUKAI}}\\
	\hfill}

\date{\today}
\maketitle
\newpage
\begin{center}
	{\huge \textbf{算数表达式求值演示}}
\end{center}
	\section{需求分析}
	\noindent \textbf{问题描述:}表达式计算式实现程序设计语言的基本问题之一,也是栈的应用的一个典型例子。设计一个程序,演示用算符优先法对算术表达式求值的过程。\\
	
	\noindent \textbf{基本要求:}以字符序列的形式从终端输入语法正确、不含变量的整数表达式。利用课件给出的算符优先关系,实现对算术四则混合运算表达式的求值,并演示在求值过程中操作符栈、操作数栈、输入字符和主要操作的变化过程。\\
	
	\noindent \textbf{输入形式:}一个以“$=$“结尾的中缀算术表达式。运算符包括$+$,$-$,$*$,$/$,$($,$)$,$=$,$\^$,参加运算的数为正整数。例如,\mintinline{C}|3*(7-2)| 或者\mintinline{C}|1024/(20+8)|或者 \mintinline{C}|(20+2)*(6*(2+8))| 或者\mintinline{C}|((2+4)*(5+7)*9+1)/2|等等。\\
	
	\noindent \textbf{输出形式:}表达式的计算结果。运算过程中操作符栈、操作数栈、输入字符和主要操作的内容。
	\newpage
	\section{概要设计}
	\noindent 中缀表达式的求值需要用到两个栈:操作符栈\mintinline{C}|OPTR|,操作数栈\mintinline{C}|OPND|。操作数栈存放的数据为\mintinline{C}|double|类型,而操作符栈存放的数据为\mintinline{C}|char|类型。
	\subsection{抽象数据类型} 
	\noindent 在本程序中,用到的\textbf{抽象数据类型}定义如下所示:\\
	
	\noindent ADT \mintinline{C}|char_stack|$\{$
	
	数据对象:$D=\{c | c\in \mintinline{C}|char|\}$
	
	数据关系:$R=\{ \left\langle e_1,e_2\right\rangle | e_1\text{是一个}\mintinline{C}|char[]|\text{类型的数组,}e_2\text{是一个指向栈顶的指针}\}$
	
	基本操作:
	
	\qquad \mintinline{C}|InitCharStack()| 操作结果:创建一个字符串类型的栈
	
	\qquad \mintinline{C}|CharStackPush(s, value)| 操作结果:入栈字符s
	
	\qquad \mintinline{C}|CharStackPop(s)| 操作结果:将栈顶元素弹出
	
	\qquad \mintinline{C}|CharStackPrint(s)| 操作结果:打印当前栈中所有元素
	
	\noindent $\}$
	
	\noindent ADT \mintinline{C}|stack|$\{$
	
	数据对象:$D=\{c | c\in \mintinline{C}|double|\}$
	
	数据关系:$R=\{ \left\langle e_1,e_2\right\rangle | e_1\text{是一个}\mintinline{C}|double[]|\text{类型的数组,}e_2\text{是一个指向栈顶的指针}\}$
	
	基本操作:
	
	\qquad \mintinline{C}|InitStack()| 操作结果:创建一个字符串类型的栈
	
	\qquad \mintinline{C}|StackPush(s, value)| 操作结果:入栈字符s
	
	\qquad \mintinline{C}|StackPop(s)| 操作结果:将栈顶元素弹出
	
	\qquad \mintinline{C}|StackPrint(s)| 操作结果:打印当前栈中所有元素
	
	\noindent $\}$
	\subsection{主程序流程}
	\noindent
	\begin{enumerate}
		\item  首先创建操作符栈\mintinline{C}|OPTR|和操作数栈\mintinline{C}|OPND|
		\item 使用\mintinline{C}|gets()|函数获取终端输入的字符串
		\item 调用自创建函数\mintinline{C}|InfixCalculation(str[], OPND, OPTR)|计算中缀表达式
	\end{enumerate}
	\newpage
	\section{详细设计}
	\noindent 两种栈的基本操作在附录中有详细代码实现。这里主要论述中缀表达式计算程序中\textbf{三个核心函数的详细设计}。
	\subsection{中缀表达式计算函数}
	\noindent 中缀表达式计算函数的定义如下:
	\begin{center}
		\mintinline{C}|void InfixCalculation(char InfixExpression[], stack *OPND, char_stack *OPTR);|
	\end{center}
	其中\mintinline{C}|char InfixExpression[]|是终端输入的中缀表达式;\mintinline{C}|stack *OPND|是操作数栈;\mintinline{C}|stack *OPTR|是操作符栈。算法思想如下:
	\begin{enumerate}
		\item 建立并初始化\mintinline{C}|OPTR|栈和\mintinline{C}|OPND|栈,然后在\mintinline{C}|OPTR|栈中压入一个\mintinline{C}|=|
		\item 扫描中缀表达式,取一字符送入\mintinline{C}|c|
		\item 当\mintinline{C}|c ==‘=’且OPTR栈的栈顶 =‘=’|时才停止循环,在\mintinline{C}|OPTR|栈的栈顶得到运算结果,否则执行以下操作:
		\begin{enumerate}
			\item 若\mintinline{C}|c|是操作数,进\mintinline{C}|OPND|栈,从中缀表达取读下一字符送入\mintinline{C}|c|
			\item 若\mintinline{C}|c|是操作符,比较\mintinline{C}|icp(ch)|的优先级(栈外优先数)和\mintinline{C}|isp(OPTR)|的优先级(栈内优先数):
			\begin{enumerate}
				\item 若\mintinline{C}|icp(ch) > isp(OPTR)|,则\mintinline{C}|c|进\mintinline{C}|OPTR|栈,从中缀表达式取下一字符送入\mintinline{C}|c|
				\item 若\mintinline{C}|icp(ch) < isp(OPTR)|,则从\mintinline{C}|OPND|栈退出两个元素\mintinline{C}|a2|和\mintinline{C}|a1|,从\mintinline{C}|OPTR|栈退出\mintinline{C}|X|,形成运算指令\mintinline{C}|(a1)X(a2)|,结果进\mintinline{C}|OPND|栈;
				\item 若\mintinline{C}|icp(c) = isp(OPTR)| 且\mintinline{C}|c == ')'|,则从\mintinline{C}|OPTR|栈退出\mintinline{C}|c == '('|,对消括号,然后从中缀表达式取下一字符送入\mintinline{C}|c|
			\end{enumerate}
			\textbf{其中,算数操作符的优先级表如下表所示}
			\begin{center}
				\begin{table}[H]
					\centering
					\begin{tabular}{c|cc}
						\hline
						操作符\mintinline{C}|c|&栈内优先数isp&栈外优先数icp\\
						\hline
						$\#$ & 0 & 0\\
						( & 1 & 6\\
						*,/,\% & 5 & 4\\
						+,- & 3 & 2\\
						) & 6 & 1\\
						\hline
					\end{tabular}
					\caption{算数操作符优先级表}
					\label{优先级}
				\end{table}
			\end{center}
		\end{enumerate}
	\end{enumerate}
	代码如下所示:
	\begin{minted}[mathescape,linenos,numbersep=5pt,gobble=2,frame=lines,framesep=2mm]{C++}
		void InfixCalculation(char InfixExpression[], stack *OPND, char_stack *OPTR){
			CharStackPush(OPTR, '=');//先往OPTR栈中压入一个#
			while(1){//扫描中缀表达式的各个字符
				if (IsOperand(InfixExpression[i]) == TRUE){//如果扫描到操作数
					double num = Digitizer(InfixExpression);//Digitizer是将字符转换为数字的函数
					StackPush(OPND, num);//将该操作数入栈
					std::cout << "当前操作符栈为:";
					CharStackPrint(OPTR);
					std::cout << "当前操作数栈为:";
					StackPrint(OPND);
					std::cout << std::endl;
				}
				else if (IsOperator(InfixExpression[i]) == TRUE){//如果扫描到操作符
					JudgePriority(InfixExpression[i],OPND,OPTR);
					//JudgePriority是用于判断当前操作符和栈顶操作符优先级的函数
				}
				if (OPTR->data[OPTR->top] == '=' && InfixExpression[i] == '=') break;//终止循环的条件
				i++;//移动到下一个字符的位置
			}
			std::cout << OPND->data[OPND->top];//栈顶元素即为中缀表达式的计算值
		}
	\end{minted}
	\subsection{字符-数字转换函数}
	\noindent 字符-数字转换函数的定义如下
	\begin{center}
		\mintinline{C}|int Digitizer(char str[]);|
	\end{center}
	其中\mintinline{C}|str[]|表示输入的中缀表达式。函数的功能是将扫描的操作数(\mintinline{C}|char|形式)转换为数字(\mintinline{C}|double|形式)。由于\textbf{限定了参加运算的数为正整数,因此考虑下述两种情况:}
	\begin{enumerate}
		\item 运算数为正的一位数,如0,1,...。在中缀表达式中,这样的运算数后必然是运算符
		\item 运算数为正的多位数,如1024。
	\end{enumerate}
	我们可以这样区分一位数与多位数:首先设置标记变量\mintinline{C}|sign = 0|。当扫描到中缀表达式的某个单字符\mintinline{C}|str[i]|时,我们将其转化为数字\mintinline{C}|x|,并置标记\mintinline{C}|sign = 1|。在扫描下一个单字符时,如果仍然扫到数字,则表明运算数为多位数,通过
	\begin{center}
		\mintinline{C}|x = x * 10 + (double)str[i+1]|
	\end{center}
	实现位数的拼接;如果扫到操作符,则表明运算数为单位数,\mintinline{C}|Digitizer(str)|函数直接返回\mintinline{C}|x|值。\\
	
	\noindent 代码如下所示:
	\begin{minted}[mathescape,linenos,numbersep=5pt,gobble=2,frame=lines,framesep=2mm]{C++}
		int Digitizer(char str[]){
			double x = 0;//要返回的数字
			int sign = 0;//标识变量,用于将字符串转换为十位数、百位数
			while (IsOperand(str[i]) == TRUE){
				if (sign != 1){
					x = char_to_double(str[i]);
					sign = 1;
					//把这个字符转换为数字后,置sign = 1
					//如果这是一位数,则str[i+1]必然不是数字
					//如果这是多位数,则str[i+1]仍然是数,由下面的语句将各个位数拼接成多位数
				}
				else{
					x = x * 10 + char_to_double(str[i]);
				}
				i++;
			}
			i--;
			return x;
		}
	\end{minted}
	\subsection{优先级判断函数}
	\noindent 优先级判断函数的定义如下
	\begin{center}
		\mintinline{C}|void JudgePriority (char c,stack *OPND, char_stack *OPTR);|
	\end{center}
	该函数用于判断当前算数操作符\mintinline{C}|c|和栈顶操作符\mintinline{C}|OPTR->data[OPTR->top]|优先级的函数。该函数依托表\ref{优先级}实现,代码如下所示:
	\begin{minted}[mathescape,linenos,numbersep=5pt,gobble=2,frame=lines,framesep=2mm]{C++}
		void JudgePriority (char c,stack *OPND, char_stack *OPTR){
			//根据算数运算符优先级表制作以下对应法则
			//OPND是操作数栈
			//OPTR是操作符栈
			int isp = 0;//栈内优先数
			int icp = 0;//栈外优先数
			if (c == '=') icp = 0;
			else if (c == '(') icp = 6;
			else if (c == '*' || c == '/' || c == '%') icp = 4;
			else if (c == '+' || c == '-') icp = 2;
			else icp = 1;
			
			if (OPTR->data[OPTR->top] == '=') isp = 0;
			else if (OPTR->data[OPTR->top] == '(') isp = 1;
			else if (OPTR->data[OPTR->top] == '*' || OPTR->data[OPTR->top] == '/' ||
				OPTR->data[OPTR->top] == '%') isp = 5;
			else if (OPTR->data[OPTR->top] == '+' || OPTR->data[OPTR->top] == '-') isp = 3;
			else isp = 6;
			
			if (icp > isp){
				//如果icp(c) > isp(OPTR),则c进OPTR栈
				//从中缀表达式中取下一字符送入c
				CharStackPush(OPTR, c);
				std::cout << "当前操作符栈为:";
				CharStackPrint(OPTR);
				std::cout << "当前操作数栈为:";
				StackPrint(OPND);
				std::cout << std::endl;
			}
			else if (icp < isp){
				//如果icp(c) < isp(OPTR)
				//OPND栈退出a2和a1,从OPTR栈退出θ,形成运算指令(a1)θ(a2)
				//结果压入OPND栈
				double a2 = StackTop(OPND);
				StackPop(OPND);
				double a1 = StackTop(OPND);
				StackPop(OPND);
				char this_OPTR = CharStackTop(OPTR);
				double ans = Calculate(a1, a2 ,this_OPTR);
				std::cout << "当前执行操作:" << a1 << this_OPTR << a2 << '=' << ans << std::endl;
				StackPush(OPND,ans);//运算结果压入OPND栈
				CharStackPop(OPTR);//运算符θ退出OPTR栈
				std::cout << "当前操作符栈为:";
				CharStackPrint(OPTR);
				std::cout << "当前操作数栈为:";
				StackPrint(OPND);
				std::cout << std::endl;
				
				JudgePriority(c, OPND, OPTR);
			}
			else{
				//果icp(c) = isp(OPTR),有两种情况:
				//如果c = '=' 则整个运算结束
				//如果c = ')' 则需要从OPTR中弹出'('从而对消括号
				if (OPTR->data[OPTR->top] == '=') return;
				else{
					CharStackPop(OPTR);
					std::cout << "当前操作符栈为:";
					CharStackPrint(OPTR);
					std::cout << "当前操作数栈为:";
					StackPrint(OPND);
					std::cout << std::endl;
				}
			}
		}
	\end{minted}
	\textbf{注意到}对于\mintinline{C}|else if (icp < isp)|所在的分支,当栈外优先级$<$栈内优先级时,需要不断将\mintinline{C}|OPTR|中的运算符弹出,直到栈外优先级$\ge$栈内优先级,才能读入下一个操作符。\\
	因此,我们采用\textbf{递归调用}的处理方法,递归调用\mintinline{C}|JudgePriority(c,OPND,OPTR)|,直到\mintinline{C}|icp = isp|或\mintinline{C}|icp > isp|才结束递归,读入新的操作符\mintinline{C}|c|。
	\section{用户使用说明}
	\noindent 程序运行后,在终端输入中缀算数表达式。输入格式有以下要求:
	\begin{itemize}
		\item 中缀表达式必须以\mintinline{C}|=|结尾。
		\item 中缀表达式间不得有“非运算符”、“非运算数”,如空格、$\#$、$\&$等。\mintinline{C}|114514 * (1919 - 810) = |是不合法的输入。
		\item 中缀表达式必须是可计算的。\mintinline{C}|(((2+4)*(5+7)*9+1)/2=|是不合法的输入,因为左侧多了一个\mintinline{C}|(|。
		\item 中缀表达式中的运算数必须为正整数。不能输入变量(如$a,b,e$),负数,小数等。
	\end{itemize}
	\newpage
	\section{测试结果}
	\noindent 有以下测试样例供参考:
	\begin{align*}
		&\text{输入:}\mintinline{C}|3*(7-2)=|&\qquad \text{输出:15}\\
		&\text{输入:}\mintinline{C}|1024/(20+8)=|&\qquad \text{输出:36.5714}\\
		&\text{输入:}\mintinline{C}|(20+2)*(6*(2+8))=|&\qquad \text{输出:1320}\\
		&\text{输入:}\mintinline{C}|((2+4)*(5+7)*9+1)/2=|&\qquad \text{输出:324.5}
	\end{align*}
	以第四组测试结果为例:
	\begin{minted}[mathescape,linenos,numbersep=5pt,gobble=2,frame=lines,framesep=2mm]{C++}
			当前操作符栈为:= ( 
			当前操作数栈为:
			
			当前操作符栈为:= ( ( 
			当前操作数栈为:
			
			当前操作符栈为:= ( ( 
			当前操作数栈为:2 
			
			当前操作符栈为:= ( ( + 
			当前操作数栈为:2 
			
			当前操作符栈为:= ( ( + 
			当前操作数栈为:2 4 
			
			当前执行操作:2+4=6
			当前操作符栈为:= ( ( 
			当前操作数栈为:6 
			
			当前操作符栈为:= ( 
			当前操作数栈为:6 
			
			当前操作符栈为:= ( * 
			当前操作数栈为:6 
			
			当前操作符栈为:= ( * ( 
			当前操作数栈为:6 
			
			当前操作符栈为:= ( * ( 
			当前操作数栈为:6 5 
			
			当前操作符栈为:= ( * ( + 
			当前操作数栈为:6 5 
			
			当前操作符栈为:= ( * ( + 
			当前操作数栈为:6 5 7 
			
			当前执行操作:5+7=12
			当前操作符栈为:= ( * ( 
			当前操作数栈为:6 12 
			
			当前操作符栈为:= ( * 
			当前操作数栈为:6 12 
			
			当前执行操作:6*12=72
			当前操作符栈为:= ( 
			当前操作数栈为:72 
			
			当前操作符栈为:= ( * 
			当前操作数栈为:72 
			
			当前操作符栈为:= ( * 
			当前操作数栈为:72 9 
			
			当前执行操作:72*9=648
			当前操作符栈为:= ( 
			当前操作数栈为:648 
			
			当前操作符栈为:= ( + 
			当前操作数栈为:648 
			
			当前操作符栈为:= ( + 
			当前操作数栈为:648 1 
			
			当前执行操作:648+1=649
			当前操作符栈为:= ( 
			当前操作数栈为:649 
			
			当前操作符栈为:= 
			当前操作数栈为:649 
			
			当前操作符栈为:= / 
			当前操作数栈为:649 
			
			当前操作符栈为:= / 
			当前操作数栈为:649 2 
			
			当前执行操作:649/2=324.5
			当前操作符栈为:= 
			当前操作数栈为:324.5 
	\end{minted}
	\newpage
	\section{调试分析}
	\subsection{算法的时空分析}
	\noindent \textbf{1.时间复杂度:}
	
	显然,我们必须遍历中缀表达式的每个元素才能计算出中缀表达式的值,故共执行$n$次操作,其中$n$是中缀表达式的长度。
	
	对于每个中缀表达式的元素,即使在最坏的情况下,\textbf{我们执行的一系列操作的时间复杂度都是常数级的}(如:入栈运算符、入栈运算数、弹出运算符执行有限次、字符-数字转换执行有限次等)。因此,整个算法的时间复杂度是$O(n)$。\\
	
	\noindent \textbf{2.空间复杂度:}
	
	我们一共使用了两个栈\mintinline{C}|OPTR,OPND|,栈的最大大小取决于中缀表达式的结构。在最坏情况下,栈的大小可能与中缀表达式的长度成正比。因此,整个算法的空间复杂度是$O(n)$。
	
	\newpage
	\section{附录}
	\noindent 详细的代码实现如下所示。也可以通过\url{https://github.com/GONGSHUKAI/Data_Structure/tree/main/Lab_Code/Lab_1/Sept.22_Lab}下载代码原文件。
	\begin{minted}[mathescape,gobble=2,frame=lines,framesep=2mm]{C++}
		#include <iostream>
		#include <cmath>
		//#define OVERFLOW -2
		//#define UNDERFLOW -1
		#define TRUE 1
		#define FALSE 0
		#define MAX 100//栈的大小
		#define MAXLENGTH 100//中缀表达式的大小
		
		typedef struct char_stack{
			char data[MAX];
			int top;
		}char_stack;
		
		typedef struct stack{
			double data[MAX];
			int top;
		}stack;
		
		int i = 0;
		
		char_stack* InitCharStack();
		void CharStackPush(char_stack *s, char value);//入栈
		void CharStackPop(char_stack *s);//出栈
		void CharStackPrint(char_stack *s);//打印当前栈内所有元素
		int IsOperand (char s);//判断是否为操作数
		int IsOperator (char s);//判断是否为操作符
		void JudgePriority (char c,stack *OPND, char_stack *OPTR);//判断算数操作符的优先级
		int Digitizer(char str[]);
		
		stack* InitStack();
		void DestroyStack(stack *s);
		int StackEmpty(stack *s);
		double StackTop(stack *s);
		void StackPush(stack *s, double value);
		void StackPop(stack *s);
		int StackSize(stack *s);
		void StackPrint(char_stack *s);//打印当前栈内所有元素
		double char_to_double(char s);
		double Calculate (double x, double y, char s);
		
		void InfixCalculation(char InfixExpression[], stack *OPND, char_stack *OPTR);
		
		int main(){
			stack *OPND = InitStack();
			char_stack *OPTR = InitCharStack();
			
			char InfixExpression[MAXLENGTH];
			gets(InfixExpression);
			InfixCalculation(InfixExpression, OPND, OPTR);
		}
		
		stack* InitStack(){
			stack *s = new stack;
			s->top = -1;//栈顶指针(即数组下标)赋初值-1
			return s;
		}
		
		//栈底下标是0,栈顶下标最多到MAX-1
		void DestroyStack(stack *s){
			delete s;
		}
		
		int StackEmpty(stack *s){//如果栈空返回1,否则返回0
			if (s->top == -1) return TRUE;
			else return FALSE;
		}
		
		double StackTop(stack *s){//返回栈顶元素
			if (s->top > -1 && s->top < MAX) return s->data[s->top];
			else return FALSE;
		}
		
		void StackPush(stack *s, double value){//压入栈
			if (s->top == MAX - 1) return;
			else s->data[++s->top] = value;
		}
		
		void StackPop(stack *s){//弹出栈顶元素
			if (s->top == -1) return;
			else{
				s->top--;
			}
		}
		
		int StackSize(stack *s){//返回栈的大小
			return s->top++;
		}
		
		void StackPrint(stack *s){//打印当前栈内所有元素
			for (int i = 0 ; i <= s->top ; i++){
				std::cout << s->data[i] << " ";
			}
			std::cout << std::endl;
		}
		
		double char_to_double(char s){
			if (s == '0') return 0;
			else if (s == '1') return 1;
			else if (s == '2') return 2;
			else if (s == '3') return 3;
			else if (s == '4') return 4;
			else if (s == '5') return 5;
			else if (s == '6') return 6;
			else if (s == '7') return 7;
			else if (s == '8') return 8;
			else return 9;
		}
		
		double Calculate (double x, double y, char s){
			if (s == '+') return x+y;
			else if (s == '-') return x-y;
			else if (s == '*') return x*y;
			else if (s == '/') return x/y;
			else return std::pow(x,y);
		}
		
		char_stack* InitCharStack(){
			char_stack *s = new char_stack;
			s->top = -1;//栈顶指针(即数组下标)赋初值-1
			return s;
		}
		
		char CharStackTop (char_stack *s){
			if (s->top > -1 && s->top < MAX) return s->data[s->top];
			else return FALSE;
		}
		
		void CharStackPush(char_stack *s, char value){//压入栈
			if (s->top == MAX - 1) return;
			else s->data[++s->top] = value;
		}
		
		void CharStackPop(char_stack *s){
			if (s->top == -1) return;
			else{
				s->top--;
			}
		}
		
		void CharStackPrint(char_stack *s){//打印当前栈内所有元素
			for (int i = 0 ; i <= s->top ; i++){
				std::cout << s->data[i] << " ";
			}
			std::cout << std::endl;
		}
		
		int IsOperand (char s){
			if (s != '(' && s != ')' && s != '+' && s != '-' && 
			    s != '*' && s != '/' && s != '%' && s != '^' && s != '='){
				return TRUE;
			}
			else return FALSE;
		}
		
		int IsOperator (char s){
			if (s == '(' || s == ')' || s == '+' || s == '-' ||
			    s == '*' || s == '/' || s == '%' || s == '^' ||s == '='){
				return TRUE;
			}
			else return FALSE;
		}
		
		int Digitizer(char str[]){
			double x = 0;//要返回的数字
			int sign = 0;//标识变量,用于将字符串转换为十位数、百位数
			int negative_sign = 0;//标识变量,用于将字符串转换为负数
			while (IsOperand(str[i]) == TRUE){
				if (sign != 1){
					x = char_to_double(str[i]);
					sign = 1;
					//把这个字符转换为数字后,置sign = 1
					//如果这是一位数,则str[i+1]必然不是数字
					//如果这是多位数,则str[i+1]仍然是数,由下面的语句将各个位数拼接成多位数
				}
				else{
					x = x * 10 + char_to_double(str[i]);
				}
				i++;
			}
			i--;
			return x;
		}
		
		void JudgePriority (char c,stack *OPND, char_stack *OPTR){
			//s1是操作数栈
			//s2是操作符栈
			int isp = 0;//栈内优先数
			int icp = 0;//栈外优先数
			if (c == '=') icp = 0;
			else if (c == '(') icp = 8;
			else if (c == '^') icp = 6;
			else if (c == '*' || c == '/' || c == '%') icp = 4;
			else if (c == '+' || c == '-') icp = 2;
			else icp = 1;
			
			if (OPTR->data[OPTR->top] == '=') isp = 0;
			else if (OPTR->data[OPTR->top] == '(') isp = 1;
			else if (OPTR->data[OPTR->top] == '^') isp = 7;
			else if (OPTR->data[OPTR->top] == '*' || OPTR->data[OPTR->top] == '/' || OPTR->data[OPTR->top] == '%') isp = 5;
			else if (OPTR->data[OPTR->top] == '+' || OPTR->data[OPTR->top] == '-') isp = 3;
			else isp = 8;
			
			if (icp > isp){
				//如果icp(c) > isp(OPTR),则c进OPTR栈
				//从中缀表达式中取下一字符送入c
				CharStackPush(OPTR, c);
				std::cout << "当前操作符栈为:";
				CharStackPrint(OPTR);
				std::cout << "当前操作数栈为:";
				StackPrint(OPND);
				std::cout << std::endl;
			}
			else if (icp < isp){
				//如果icp(c) < isp(OPTR)
				//OPND栈退出a2和a1,从OPTR栈退出θ,形成运算指令(a1)θ(a2)
				//结果压入OPND栈
				double a2 = StackTop(OPND);
				StackPop(OPND);
				double a1 = StackTop(OPND);
				StackPop(OPND);
				char this_OPTR = CharStackTop(OPTR);
				double ans = Calculate(a1, a2 ,this_OPTR);
				std::cout << "当前执行操作:" << a1 << this_OPTR << a2 << '=' << ans << std::endl;
				StackPush(OPND,ans);//运算结果压入OPND栈
				CharStackPop(OPTR);//运算符θ退出OPTR栈
				std::cout << "当前操作符栈为:";
				CharStackPrint(OPTR);
				std::cout << "当前操作数栈为:";
				StackPrint(OPND);
				std::cout << std::endl;
				
				JudgePriority(c, OPND, OPTR);
			}
			else{
				//果icp(c) = isp(OPTR),有两种情况:
				//如果c = '=' 则整个运算结束
				//如果c = ')' 则需要从OPTR中弹出'('从而对消括号
				if (OPTR->data[OPTR->top] == '=') return;
				else{
					CharStackPop(OPTR);
					std::cout << "当前操作符栈为:";
					CharStackPrint(OPTR);
					std::cout << "当前操作数栈为:";
					StackPrint(OPND);
					std::cout << std::endl;
				}
			}
		}
		
		
		
		
		void InfixCalculation(char InfixExpression[], stack *OPND, char_stack *OPTR){
			CharStackPush(OPTR, '=');
			while(1){
				if (IsOperand(InfixExpression[i]) == TRUE){
					double num = Digitizer(InfixExpression);
					//double num = char_to_double(InfixExpression[i]);
					StackPush(OPND, num);
					std::cout << "当前操作符栈为:";
					CharStackPrint(OPTR);
					std::cout << "当前操作数栈为:";
					StackPrint(OPND);
					std::cout << std::endl;
				}
				else if (IsOperator(InfixExpression[i]) == TRUE){
					JudgePriority(InfixExpression[i],OPND,OPTR);
				}
				if (OPTR->data[OPTR->top] == '=' && InfixExpression[i] == '=') break;
				i++;
			}
			std::cout << OPND->data[OPND->top];
		}
	\end{minted}
\end{document}